\section{$k$-Sketch Query}\label{sec:related_work:sketch}
Our proposed $k$-Sketch query is closely related
to the following four areas: 
\emph{news discovery in computational journalism}, 
\emph{frequent episode mining on sequence data}, 
\emph{top-$k$ diversity query} and \emph{event detection in information retrieval}.
%automatic news detection, frequent episode mining and top-$k$ diversity query.

\subsection{Computational Journalism}
An important aspect of computational journalism
is to leverage computational technology to discover
striking news themes. 
Previous works on automatic news theme generation belong to two categories: \emph{dimension-oriented} approach and \emph{subject-oriented} approach. Dimension-oriented approaches aim to select appropriate dimensions to make an event interesting. 
Representative works include the \emph{situational facts}~\cite{sultana2014incremental} and the \emph{one-of-the-few facts}~\cite{wu2012one}. On the other hand, subject-oriented approaches aim to
summarize subjects' histories from their historical events, such as the \emph{prominent streaks}~\cite{zhang2014discovering}. Our proposed solution falls into the subject-oriented category. 


\subsubsection{Situational facts and One-of-the-few facts}
\textbf{Situational Facts~\cite{sultana2014incremental}:} it finds for a given event, the best constraint-measure pair that makes the event unique (i.e., not dominated by others). 
An example of the \emph{situational fact} is listed in Table~\ref{tbl:sketh:related_works} and is extracted from the NBA game events where the measure dimensions are ``points, assists, and rebounds'' and the constraint dimensions are ``team, result, and date''. The \emph{situational fact} of the event in Table~\ref{tbl:sketh:related_works} is the constraint-measure pair $\langle$team=Blazer, points$\rangle$, since under
this situation, this event is a skyline among all events (i.e., no other events matching the constraint ``team=Blazer'' contain an even higher ``point'').


\noindent\textbf{One-of-the-few Facts~\cite{wu2012one}:}  it finds the dimensions
under which no more than $\tau$ events are in the $k$-skyband (i.e., not dominated by $k$ events and $k$ is as small as possible).
An example of the \emph{one-of-the-few fact} is listed
in Table~\ref{tbl:sketh:related_works}. In the example, when $\tau=5$, two dimensions are selected (i.e., points and rebounds). Under these two dimensions, five players are the skylines (i.e., $1$-skyband), thus each of them is a \emph{one-of-the-5} player.

As demonstrated, ``situational facts''
and ``one-of-the-few facts'' are dimension-oriented since they attempt to generate news themes by selecting dimensions.


\begin{table}[h]
\centering
\caption{Examples of different news themes.}
\label{tbl:sketh:related_works}
\begin{tabular}{|l|p{10cm}|}
\hline 
\textbf{Method} & \textbf{Example news theme}\\
\hline
Situational facts~\cite{sultana2014incremental} & Damon Stoudamire scored 54
points on January 14, 2005. It is the highest score in history made
by any Trail Blazer. \\
\hline
One-of-the-$\tau$ facts~\cite{wu2012one} & Jordan, Chamberlain, James, Baylor and Pettit are the five players with highest points and rebounds in NBA history. \\
\hline
Prominent streaks~\cite{zhang2014discovering} & \makecell[l]{ 1.Kobe scored 40+ in 9 straight games,\\ first in his career! \\  2.Kobe scored 50+ in 4 straight games, \\first in his career!\\  3...} \\
\hline
$k$-Sketch & \makecell[l]{1.Kobe scored 40+ in 9 straight games, \\ ranked 4th in NBA history! \\  2.Kobe scored 50+ in 4 straight games, \\ ranked 1st in NBA history.  \\ 3....} \\
\hline
\end{tabular}
\end{table}

\subsubsection{Prominent streaks}
Zhang et al.~\cite{zhang2014discovering} proposed a
subject-oriented approach to generate news themes by discovering \emph{prominent streaks}. 
In~\cite{zhang2014discovering}, a streak is modeled as a pair of the streak duration and the minimum
value of all events in the streak.
For example, as shown in Table~\ref{tbl:sketh:related_works}, a streak of ``Kobe'' may be $\langle$9 consecutive games$,$ minimum points of 40$\rangle$.
The objective of~\cite{zhang2014discovering}
is to discover all the non-dominated streaks (i.e., prominent streaks) where the dominance is defined among streaks of the same subject. 
Despite being the same subject-oriented approach,
our $k$-Sketch query differs from \cite{zhang2014discovering} in two aspects. First, we look at the \emph{global prominence among all} subjects (i.e., rank in NBA history) rather than local prominence within one subject (i.e., non-dominance in one's career). 
Second, our model provides the best $k$ ranked-streaks
for each subject, whereas~\cite{zhang2014discovering} returns a set of skylines 
which could be potentially large.

 
\subsection{Frequent Episode Mining}
In sequence data mining, an episode~\cite{mannila1997discepisodes,
zhou2010serialepisodes, tatti2012strictepisodes, laxman2007nonoverlapepisodes} is 
defined as a collection of time sequenced events which occur together within a time window. The uniqueness of an episode is determined by the containing events. The objective of frequent episode mining is to discover episodes whose 
occurrences exceed a threshold.  
Our $k$-Sketch query differs from the episode mining in two major aspects. 
First, episodes are associated with categorical values thus they can be grouped to count the occurrences. 
On the other hand, our ranked-streaks are defined with numerical values, making it inappropriate to be grouped.
Second, the episodes are selected based on the occurrences 
which do not contain the \emph{rank} information, whereas our 
$k$-Sketch query explicitly provides the rank of selected streaks.  
As such, episode mining techniques cannot support $k$-Sketch query.

\subsection{Top-$k$ Diversity Query}
Top-$k$ diversity queries~\cite{agrawal2009diversifying,borodin2012max,drosou2014diverse,chen2015diversity}
aim to find a subset of objects to maximize a scoring function. The scoring function normally penalizes
subsets with similar elements. Our 
$k$-Sketch query has two important distinctions.
First, the inputs to top-$k$ diversity queries are known in advance, whereas in $k$-Sketch query, the ranks of streaks need to be derived. 
Second, existing methods for online diversity queries~\cite{borodin2012max,drosou2014diverse,chen2015diversity} only study
the update on a single result set when a new event arrives. However, our online sketch maintenance 
incurs the problem of multiple sketch updates for each new event. Such a complex update pattern has not been studied yet.

\subsection{Event Detection and Tracking}
In information retrieval, event detection and tracking aim to extract and organize new events from various
media sources such as text streams~\cite{allan1998line,brants2003asystem}, social media streams~\cite{li2015social} and web articles~\cite{Vuurens2015Onlinenews}. 
Despite the usefulness of these works, they differ from our $k$-Sketch query as they focuses on the detection of a single event, whereas $k$-Sketch aims to summarize a subject's history.
Therefore, the abovementioned techniques cannot be directly applied.

