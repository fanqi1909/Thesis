\chapter{Introduction}
With the maturity of database technologies, many applications today collect and store data from all domains 
at unprecedented scale. For example, billions of 
social network users and their activities are collected in the form
of \emph{graphs}; Thousands of sensor ticks are collected every second
in the form of \emph{time series}; Hundreds of millions of spatiotemporal points
are collected as \emph{trajectories}.
Flooded by the tremendous amount of data, 
it is becoming increasingly critical to efficiently
conduct analytics to discover useful insights.
However, traditional SQL analytics, which comprises 
primary operations (e.g., partition, sorting and aggregation),
is unable to cope  with domain-specific analytics 
such as graph traversal and pattern detection. 
In practice, expressing these domain-specific 
analytics using SQL query often involves 
complex joins which are hard to optimize. 
In this thesis, we introduce the concept of \emph{neighborhood analytics} which
originates from the recognized SQL window functions. We study
the usage of \emph{neighborhood analytics} in different data domains
and design efficient algorithms to cope with today's big data.

%In this thesis, we explore an important type of 
%SQL analytics named \emph{window function}
%in supporting domain-specific analytics. 
%To this end, we generalize
%the window function to the \emph{neighborhood analytics} and propose
%three particular queries of such analytics in different domains.
%
%neighborhood-based domain-specific queries.
%Furthermore, we address the technical issues on how to
%efficiently process these neighborhood-based queries with regard to nowadays data sizes.
% efficiently
%process these neighborhood-based queries.

%which is expressive
%in many data domains. Furthermore, we address the technical issues on 

\section{Neighborhood Analytic}
By its self-describing name, neighborhood analytic aims to provide
summaries of each object over its vicinity. In contrast to the global
analytics which aggregates the entire collection of data as a whole, neighborhood
analytic provides a personalized view on each object per se. Neighborhood
data analytics originates from the window function defined in  SQL which is
illustrated in Figure~\ref{fig:window}.

\begin{figure}[h]
\centering
\includegraphics[width=0.8\linewidth]{window_example.pdf}
\caption{A SQL window function computing running sum and average of
sales. The window of season-3 is highlighted.} 
\label{fig:window}
\end{figure}

As shown in the figure, the sales report contains five attributes: 
``Season'', ``Region'' and ``Sales'' are the original fields, ``$\mathtt{sum()}$'' and ``$\mathtt{avg()}$''
are the analytics representing the running sum and average. A window function
is represented by the $\mathtt{over}$ keyword. In this context, the window of a tuple $o_i$
contains other tuples $o_j$ such that $o_i$ and $o_j$ are in the same ``region'' and $o_j$'s ``season'' is
prior to $o_i$'s. The window of season-$3$ for region-``West'' is highlighted.
Apart from this example, there are many other 
usages of the window function in the relational context. 
Being aware of the success of the window function, 
SQL~11 standard incorporates ``$\mathtt{LEAD}$'' and ``$\mathtt{LAG}$'' 
keywords which offer fine-grained specifications on a tuple's window.

Despite the usefulness, there are few works reporting the window
analytics in the non-relational domain. This may dues to the
usage of \emph{sorting} in relational windows. For example,
in Figure~\ref{fig:window},
objects need to be sorted according to ``Season'', and then the window of
each objects is implicitly formed. However, 
in non-relational context, sorting may be ambiguous and even undefined.

To generalize the window function to other domains, we propose the neighborhood
analytics in a broader context. Given a set of objects 
(such as tuples in relational domain or vertexes in graph domain),
the neighborhood analytic is a composite function
$(\mathcal{F} \circ \mathcal{N})$ applied on every object. $\mathcal{N}$
is the \emph{neighborhood function}, which contains the related objects of an object;
$\mathcal{F}$ is an \emph{analytic function}, which could be aggregate, rank,
pattern matching, etc.
Apparently, the relational window function is a special case of the
neighborhood analytics. For example, the window function in Figure~\ref{fig:window} 
can be represented as $\mathcal{N}(o_i)=\{o_j | o_i.season > o_j.season \wedge o_i.region = o_j.region\}$
and $\mathcal{F} = \mathtt{avg}$.
Since the \emph{sorting} requirement is relaxed, our neighborhood analytics is able to
enrich the semantic of relational window notations 
and can be applied on many other domains.
\section{Scope of the Thesis}
In this thesis, we explore the neighborhood analytics in
three prevalent data domains, namely \textbf{attributed graph},
\textbf{sequence data} and \textbf{trajectory}. 
To provide useful analytics, we define the following
two intuitive instances of the neighborhood function:
%Since the neighborhood analytics could be very broad,
%this thesis only focus on the following 
%two intuitive neighborhood functions:

\textbf{Distance Neighborhood}: the neighborhood is defined based on numeric distance, that is $\mathcal{N}(o_i,K) = \{o_j | \mathtt{dist}(o_i,o_j) \leq K \}$, where $\mathtt{dist}$ is a distance function and $K$ is a distance threshold.

\textbf{Comparison Neighborhood}: the neighborhood is defined based on the comparison of objects, that is $\mathcal{N}(o) = \{o_i | o.a_m \ \mathtt{cmp} \ o_i.a_m\}$, where $a_m$ is an attribute of object
and $\mathtt{cmp}$ is a binary comparator (i.e., $=,<,>,\leq,\neq,\geq$).

Despite the simplicity of these two neighborhood functions, they can weave many useful analytics as we shall see in the remaining part of the thesis.
%
%There could be other types of neighborhood functions with different constraints. 
%However, despite the simpleness of these two neighborhood functions, they 
%are indeed versatile in representing many useful analytics.

%In particular, we looked at three most prevalent data domains, namely \textbf{attributed graph},
%\textbf{time series} and \textbf{trajectory}. 
%We then categorize two intuitive neighborhood functions as follows:
%
%\textbf{Distance Neighborhood}: the neighborhood is defined based on numeric distance, that is $\mathcal{N}(o_i,K) = \{o_j | \mathtt{dist}(o_i,o_j) \leq K \}$, where $\mathtt{dist}$ is a distance function and $K$ is a distance threshold.
%
%\textbf{Comparison Neighborhood}: the neighborhood is defined based on the comparison of objects, that is $\mathcal{N}(o) = \{o_i | o.a_m \ \mathtt{cmp} \ o_i.a_m\}$, where $a_m$ is an attribute of object
%and $\mathtt{cmp}$ is a binary comparator.
%
%There could be other types of neighborhoods with more fine-grained or more general definitions. 
%In this thesis, we demonstrate
%that these two simple neighborhood definitions
%joint with traditional analytic functions are already versatile 
%in various domains. 
%\revised{They are cable to both express existing queries and devise novel emerging 
%queries that are practically useful.
%}



\section{Thesis Contributions}
In brief, the contribution of this thesis is twofold.
First, by sewing different $\mathcal{N}$s and $\mathcal{F}$s, 
several novel neighborhood based queries are proposed for 
\emph{graph}, \emph{sequence data} and \emph{trajectory} respectively. 
%three interesting
%neighborhood analytic queries are proposed for \emph{graph}, \emph{time series} 
%and \emph{trajectory} domains respectively. 
Second, this thesis
deals with the efficiency issues in deploying the corresponding analytic queries to
handle data of large scale data.
The roadmap of this thesis is shown in Figure~\ref{fig:thesis_roadmap}.
\begin{figure}[h]
\centering
\includegraphics[width=0.8\linewidth]{thesis_roadmap.pdf}
\caption{The road map of this thesis. There are three major contributions as highlighted in the center. Each contribution
is a neighborhood based analytics with different $\mathcal{N}$ and $\mathcal{F}$ as indicated by arrows.} 
\label{fig:thesis_roadmap}
\end{figure}

In a nutshell, we propose three neighborhood based queries in respective data domains. 
In \emph{graph}, we define the Graph Window Query which summarizes the vicinity of each vertex.
The query utilizes both distance neighborhood and comparison neighborhood to facilitate both
general graphs and direct acyclic graphs.
%
%two types of queries: \emph{k-hop window query} is based on the \emph{distance} neighborhood
%and \emph{topological window query} is based on the \emph{comparison} neighborhood. 
In \emph{sequenced data}, we propose a $k$-Sketch Query to summarize a subject's history. The $k$-Sketch query builds on a nested \emph{distant} and \emph{comparison} neighborhood based pattern called \emph{rank-aware streak}.
%
% we design a nested \emph{distant} and \emph{comparison} neighborhoods based pattern called \emph{rank-aware streak}, and propose a $k$-Sketch query to select the most representative rank-aware streaks.
In \emph{trajectory}, we propose a General Co-movement Pattern (GCMP) Query to discover co-moving behaviors among moving objects. The GCMP query leverage the neighborhood notion to unify existing co-moving patterns.
%
%we analyze existing movement patterns based
%on \emph{distance} and \emph{comparison} neighborhoods and unify
%the existing works with a general pattern discovery query. 
In the following parts of this section, we present our contributions in detail.


\subsection{Graph Window Queries}
The first piece of the thesis deals with neighborhood analytics
on graph data. Nowadays information network are typically
modeled as attributed graphs where the 
vertexes correspond to objects and the edges capture the
relationships between these objects. As vertexes embed a wealth
of information (e.g., user profiles in social networks), there are 
emerging demands on analyzing these data to extract useful insights. 
We propose the concept of \emph{window analytics} 
for attributed graph and identify two types of such analytics as shown in the following examples:

\textbf{$k$-hop Window:} The $k$-hop neighbors of a vertex form
its $k$-hop window. Since the $k$-hop neighbors are the most structurally
relevant vertexes to the vertex, analytics on the information from the $k$-hop window
would be beneficial. Typical analytic queries include summarizing the related connections' distribution among different companies, and computing age distribution of the related friends can be useful.
%In a social network (such as LinkedIn and Facebook etc.), users are normally modeled as vertexes and connectivity relationships are modeled as edges. In the social network scenario, it is of great interest to summarize the most relevant connections to each user such as the neighbors within $2$-hops. Some analytic queries such as summarizing the related connections' distribution among different companies, and computing age distribution of the related friends can be useful. In order to answer these queries, collecting data from every user's neighborhoods within 2-hop is necessary.
%\end{example}

\textbf{Topological Window:} The topological neighbors are defined
in the context of Directed Acyclic Graph (DAG). In DAGs,
topological neighbors are composed of all the ascendant vertexes of a vertex. 
The topological neighbors represent the most influential vertexes of a given vertex.
Since DAGs are often found in biological networks, topological window would
be helpful to analyze the the statistics of molecules of each biological proteins' pathway.
%Since the topological neighbors , analytics
%on the topological window would benefit the deeper understanding of the 
%network.
%DAGs are often found
%in biological networks and citation networks. 
% The topological neighbors
%represent the influences from the source vertex to the given vertex, thus
%analytics over topological window would be interesting.

%\begin{example}[Topological window]
%In biological networks (such as Argocyc, Ecocyc etc.), genes, enzymes and proteins are vertexes and their
%dependencies in a pathway are edges. Because these networks are directed and acyclic, 
%in order to study the protein regulating process, one may be interested to find out the statistics of molecules in each protein production pathway. For each protein,we can traverse the 
%graph to find every other molecules that are in the upstream of its pathway.
%Then we can group and count the number of genes and enzymes among those molecules.
%\end{example}

The two \emph{windows} shown in the above examples are essentially neighborhood functions defined for each vertex. Specifically, let $G=(V,E,A)$ be an attributed graph, where $V$ is the set of vertexes, $E$ is the set of edges, and each vertex $v$ is associated as a multidimensional points $a_v \in A$ called attributes.
The \emph{k-hop} window is a \emph{distance} neighborhood function, 
i.e., $\mathcal{N}_1(v,k)= \{u|\mathtt{dist}(v,u) \leq k\}$, 
which captures the vertexes that are $k$-hop nearby. 
The \emph{topological} window,  $\mathcal{N}_2(v)= \{u | u \in v.ancestor\}$,
is a \emph{comparison} neighborhood function that captures
the ancestors of a vertex in a directly acyclic graph.  The analytic function $\mathcal{F}$ is an aggregate function (sum, avg, etc.) on $A$.

Apart from demonstrating the useful use-cases on these two windows, 
we also investigate how Graph Window Query processing can be efficiently supported.  We propose
two different types of indexes: Dense Block Index (DBIndex)
and Inheritance Index (I-Index). The DBIndex and I-Index
are specially optimized to support $k$-hop window and topological
window processing. These indexes
integrate the aggregation process with partial work sharing techniques
to achieve efficient computation.
In addition, we develop space-and-performance efficient techniques
for the index construction. Notably, DBIndex saves upto 80\%
of the index construction time as compared to the state-of-the-art competitor and upto 10x
speedup in query processing. 


\subsection{$k$-Sketch Query on Sequence Data}
The second piece of this thesis explores the neighborhood analytics on sequence data. 
%In sequence data analysis, 
As part of the sequence data analysis,
summarizing a subject's history with sensational patterns
is an important and revenue-generating task in a plethora of applications 
such as computational journalism~\cite{cohen2011computational,zhang2014discovering}, automatic fact checking~\cite{hassan2014data,walenz2014finding}, and perturbation analysis~\cite{Walenz:2016:PAD:3007328.3007330}.
%
%an important and revenue-generating task is to
%summarize a subject's history using sensational patterns.
An outstanding example of such patterns is the \emph{streak}~\cite{zhang2014discovering},
which is commonly found in stock
and sports reports. For instance:

%For example, streaks may be of the following forms in stock and sports scenarios:

%Streak is commonly used in many sequenced applications,
%such as network monitoring, stock analysis
%and sport reporting. For example, streaks may be of the following forms in stock and sports scenarios:
\begin{enumerate}
	\item{[STOCK]:``Apple Inc. has an \emph{average} price of USD 115.5 in the \emph{last week}''}
	\item{[SPORTS]: ``Kobe has scored \emph{at least} 60 points in \emph{three straight games}'' }
\end{enumerate}
In general, a streak is constructed from two concepts: an \emph{aggregate function} (e.g., average, min)
applied on \emph{consecutive events} (e.g., seven days, three games). 
%Indeed, a streak an be simply modeled as a neighborhood query.
However, the streak itself does not embed the strikingness information, which is limited
to represent a sensational pattern. 
%
%
%is a neighborhood query defined for an event where the neighborhood range is the window the event belongs to.
%%
%, a \emph{streak} can be modeled as a neighborhood query
%of events, where the neighborhood function is defined as windows.
%This makes the streak as a neighborhood query of events where the neighborhood function is defined as time windows.
%
%Since the streak is aggregated based on the
%closeness of events in the time domain (i.e., time window), it can be viewed as a neighborhood query where the neighborhood function is defined by the time distance, and the analytic function can be any aggregate function.
%However, it is challenging to directly leverage streaks to discover sensational patterns because
%the strikingness of a streak is missing.
%A direct challenge in discovering phenomenal patterns using streak is
%how to quantify the strikingness of a streak.
For example, \emph{Streak 2} would not be striking if all NBA players were able to score over 60 points. 
On the contrary, knowing that most NBA players only score 20 points in a game,
\emph{Streak 2} is indeed quite striking. 
Therefore, the strikingness of a streak should be measured by comparing with other streaks.
Based on this observation, we propose a rank-aware pattern named \emph{ranked-streak} which measures the strikingness of
a streak by comparing among all streaks under same condition (i.e., streak length).

Technically, the ranked-streak can be viewed as a joint neighborhood function. 
Let $e_s(t)$ denote the event of subject $s$ at time $t$. Then the ranked-streaks are generated using neighborhood functions in a two-step manner: 

\begin{enumerate}
\item {a \emph{distance neighborhood} $\mathcal{N}_1(o_i,w)=\{o_j | o_i.t - o_j.t \leq w \}$ groups a consecutive $w$ events for each event. Let $\overline{v}$ be the aggregate value associated with $\mathcal{N}_1$, then the output of this step is a set of \emph{streak}s of the form  $n=\langle o_i, w, t, \overline{v} \rangle$.}
\item {a \emph{comparison neighborhood} $\mathcal{N}_2(n_i) = \{n_j | n_j.w = n_i.w \wedge n_i.\overline{v} \geq n_j.\overline{v} \}$ ranks a subject's streak among all other streaks with the same streak length. The result of this step is a tuple $\langle o_i, w, t, r \rangle$, where $r$ is the \emph{rank}.}
\end{enumerate}
%(1) a \emph{distance neighborhood} $\mathcal{N}_1(o_i,w)=\{o_j | o_i.t - o_j.t \leq w \}$ groups a consecutive $w$ events for each event. Let $\overline{v}$ be the aggregate value associated with $\mathcal{N}_1$, then the output of this step is a set of \emph{streak}s of the form  $n=\langle o_i, w, t, \overline{v} \rangle$.
%
%(2) a \emph{comparison neighborhood} $\mathcal{N}_2(n_i) = \{n_j | n_j.w = n_i.w \wedge n_i.\overline{v} \geq n_j.\overline{v} \}$ ranks a subject's streak among all other streaks with the same window length. The result of this step is a tuple $\langle o_i, w, t, r \rangle$, where $r$ is the \emph{rank}.
%

As the ranked-streak contains the relative position of the streak among its cohort, it provides a quantitative measure of the strikingness.
For example, the rank-aware version of \emph{Streak 2} would be ``Kobe has scored at least 60 points in three straight games, which is best in the league''.  This clearly suggests that \emph{Streak 2} is striking.
%

On the basis of the ranked-streaks, we study the problem of effectively summarizing
a subject history. We notice that, for a subject with $n$ events, there
are $O{n \choose 2}$ streaks. 
In real life, ``Kobe'' has played $1,000$ games which may produce near half million streaks.
Such a large number of streaks is too overwhelming to represent a subject's history. Hence,
we are motivated to propose a $k$-Sketch query that selects $k$ ranked-streaks which best represent
a subject's history. To find the qualified streaks, we design a novel scoring which
considers both the events covered of the streaks and the ranks of the streaks.
% under a novel scoring function. Our scoring function
%considers both the events covered of the streaks and the ranks of the streaks.

%A prominent drawback of streak is that the amount of streaks
%%Another challenge  the pattern discovery is that the amount of streaks
%for a subject is huge. For a subject with $N$ events,
%there are $O{N\choose 2}$ streaks as well as rank-aware streaks. 
%In real life, ``Kobe'' has played $1000$ games which may produce near half million streaks.
%Such a large number clearly poses difficulties for pattern discoveries.
%%for selecting phenomen streaks. 
%Nevertheless, we observe that many streaks share almost identical information. For example, when Kobe scored 81 points
%in one game, this single-event streak ranks number 1 among all streaks. Subsequently, the streaks with size 3 (i.e., 3 straight games afterwards) ranked number 10. Thus reporting this
%streak only provide redundant striking information (i.e., the 81 point game).
%Based on this observation, we
%propose a $k$-Sketch query to select the $k$ most \emph{representative streaks} via a novel scoring function that considers both strikingness and diversity of streaks.
%
%Our $k$-Sketch query is based on a novel scoring function that consider both the 
%stringness and the diversity of the streaks. 

In this thesis, we extensively study
the technical issues in processing $k$-Sketch queries in both online and offline scenarios.
In the offline scenario, we design two pruning techniques which largely
reduces the streaks enumerated in generating ranked-streaks. Then, we adopt a
$(1-1/e)$-approximate sketch selection algorithm by utilizing the 
submodularity of the $k$-Sketch query. In the online scenario, we design
an \emph{online-streak bound} to avoid evaluating many unnecessary streaks. Furthermore,
we propose a $1/8$-approximate algorithm to facilitate efficient sketch maintenance.
In the experimental study, we compare our solutions with baselines using
four real datasets, and the results demonstrate the efficiency and 
effectiveness of our proposed algorithms: the running time
achieves up to 500x speedup as compared to baseline and the quality of the
detected sketch is endorsed by the anonymous users
from Amazon Mechanical Turk\footnote{\url{https://requester.mturk.com}}. 
%
%, our techniques is confirmed by the experiments on four 
%real datasets. And the effectiveness of our $k$-Sketch query is endorsed
%by the anonymous users from Amazon Mechanical Turk\footnote{\url{https://requester.mturk.com}}.

%%
%%Besides proposing the rank-aware streak, we also technically 
%%study how to efficiently support the \emph{$k$-Sketch} query in both
%%offline and online scenarios. We propose various window-level 
%%pruning techniques to find striking candidate steaks.
%%Among those candidates, we then develop approximation
%%methods, with theoretical bounds, to discover $k$-sketches for each subject. 
%Furthermore, we conduct experiments on four real
%datasets, and the results demonstrate the efficiency and 
%effectiveness of our proposed algorithms: the running time
%achieves up to 500x speedup as compared to baseline and the quality of the
%detected sketch is endorsed by the anonymous users
%from Amazon Mechanical Turk \footnote{\url{https://requester.mturk.com}}.

\subsection{Co-Movement Pattern Query on Trajectory Data}
The third piece of the thesis studies the neighborhood
analytics in the trajectory domain. 
In trajectory analysis, an important
mining task is to discover traveling patterns among moving objects. 
A traveling pattern is often determined by the neighborhoods of moving objects. One of the prominent examples is the \emph{co-movement pattern}~\cite{li2013effective,zheng2015trajectory}.
A co-movement pattern refers to a group of moving objects traveling together for a certain period of time. A pattern is significant if the group size exceeds $M$ and the length of duration exceeds $K$. 
The group is defined based on the spatial proximity of objects, which 
can be seen as the spatial neighbors of each object.


%Table~\ref{tbl:intro-move-patterns} summarizes several popular co-movement patterns with different 
%constraints with respect to clustering in spatial proximity, consecutiveness in temporal duration
%and computational complexity. In particular, the flock~\cite{gudmundsson2006computing} and the group~\cite{wang2006grouppattern} patterns require all the objects in a group to be enclosed by a disk with radius $r$;
%whereas the convoy~\cite{jeung2008discovery}, the swarm~\cite{li2010swarm} and the platoon~\cite{li2015platoon} patterns resort to density-based spatial clustering. In the temporal dimension,
%the flock~\cite{gudmundsson2006computing} and the convoy~\cite{jeung2008discovery} require all the timestamps
%of each detected spatial group to be consecutive, which is referred to as global consecutiveness; whereas the swarm~\cite{li2010swarm} does not impose any restrictions. 
%The group~\cite{wang2006grouppattern} and the platoon~\cite{li2015platoon} adopt 
%a compromised approach by allowing arbitrary gaps between consecutive segments, 
%which is called local consecutiveness. They introduce a parameter $L$ to control the minimum length of each local consecutive segment.
%
%
%\begin{table}[h]
%\centering
%\begin{tabular}{|l|c|c|}
%\hline 
%\textbf{Pattern} & {\textbf{Proximity}} & { \textbf{Consecutiveness}} \\% & { \textbf{Time  Complexity}}\\ 
%\hline 
%flock~\cite{gudmundsson2006computing} & disk based &  global \\%& {$O(|\mathbb{O}||\mathbb{T}|\log(|\mathbb{O}|))$} \\ 
%\hline 
%convoy~\cite{jeung2008discovery} & density based &   global \\%& {$O(|\mathbb{O}|^2+|\mathbb{O}||\mathbb{T}|)$}\\ 
%\hline 
%swarm~\cite{li2010swarm} & density based  & - \\%& {$O(2^{|\mathbb{O}|}|\mathbb{O}||\mathbb{T}|)$}  \\ 
%\hline 
%group~\cite{wang2006grouppattern} & disk based &  local \\%& {$O(|\mathbb{O}|^2|\mathbb{T}|)$}\\ 
%\hline 
%platoon~\cite{li2015platoon} & density based &  local \\%& {$O(2^{|\mathbb{O}|}|\mathbb{O}||\mathbb{T}|)$}\\ 
%\hline 
%\end{tabular} 
%\caption{Constraints of co-movement patterns.}
%\label{tbl:intro-move-patterns}
%\end{table}
%Neighborhoods relationship
%among moving objects forms the traveling behavior of the objects
%which is often referred as moving patterns. Specifically, the neighborhoods
%of objects are defined based on the spatial proximity and the 
%
%
%nIn trajectory domain, an instance 
%of neighborhood usage is to detect the movement pattern. In simple
%words, a movement pattern is a set of moving object whose trajectories
%exhibits interesting behaviors.
%
%Due to its wide spectrum of applications, 
%mining the movement patterns have attracted many research attention.

%
%In trajectory domain, the neighborhoods of objects across multiple time snapshots collectively 
%form a movement pattern. Due to its wide spectrum of applications, mining the movement patterns have attracted many research attention.
%
%In the trajectory domain, neighborhood
%functions can effectively model the movement patterns
%of objects which is useful in a wide
%spectrum of applications. 
%
%An important mining task in trajectory
%data that has drawn many research attention is discovering
%movement patterns. Coincidentally, a movement pattern
%can be 
%
%
%Discovering co-movement patterns from large-scale trajectory 
%databases is an important mining task and has a wide
%spectrum of applications. 
%Existing studies have identified several types of interesting movement patterns called \emph{co-movement} patterns. A co-movement pattern refers to a group of moving objects traveling together for a certain period of time and the group of objects is normally determined by their spatial proximity. A pattern is prominent if the group size exceeds $M$ and the length of duration exceeds $K$. 
%Inspired by the basic definition 
%and driven by different mining applications, there are a bunch of variant 
%co-movement patterns that have been developed with more advanced constraints.

%\begin{figure}[h]
%\centering
%\includegraphics[width=0.8\textwidth]{trajectory_patterns.pdf}
%\caption{Trajectories and co-movement patterns. The example consists of six trajectories across six snapshots. Objects in spatial clusters are enclosed by dotted circles. $M$ is the minimum cluster cardinality; $K$ denotes the minimum number of snapshots for the occurrence of a spatial cluster; and $L$ denotes the minimum length for local consecutiveness.}
%\label{fig:related_work}
%\end{figure}
%
%Figure~\ref{fig:related_work} is an example to demonstrate the concepts of various co-movement patterns. The trajectory database consists of six moving objects and the temporal dimension is discretized into six snapshots. In each snapshot, we treat the clustering method as a black-box and assume that they generate the same clusters. Objects in proximity are grouped in the dotted circles. As aforementioned, there are three parameters to determine the co-movement patterns and the default settings in this example are $M=2$, $K=3$ and $L=2$. Both the \emph{flock} and the \emph{convoy} require the spatial clusters to last for at least $K$ consecutive  timestamps. Hence,$\langle o_3,o_4:1,2,3 \rangle$ and $\langle o_5,o_6:3,4,5 \rangle$  remains the only two candidates matching the patterns. The \textit{swarm} relaxes the pattern matching by discarding the temporal consecutiveness constraint. Thus, it generates many more candidates than the \textit{flock} and the \textit{convoy}. The \textit{group} and the \textit{platoon} add another constraint on local consecutiveness to retain meaningful patterns. For instance, $\langle o_1,o_2:1,2,4,5 \rangle$ is a pattern matching local consecutiveness because timestamps $(1,2)$ and $(4,5)$ are two segments with length no smaller than $L=2$. The difference between the \textit{group} and the \textit{platoon} is that the \textit{platoon} has an additional parameter $K$ to specify the minimum number of snapshots for the spatial clusters. This explains why $\langle o_3,o_4,o_5:2,3 \rangle$ is a \textit{group} pattern but not a \textit{platoon} pattern.


%As shown, there are various types of co-movement patterns facilitating different application needs and 

Rooted from the basic movement definition and driven by different mining applications, there are several instances of co-movement patterns that have been developed with more advanced constraints, namely \emph{flock}~\cite{gudmundsson2006computing}, \emph{convoy}~\cite{jeung2008discovery}, \emph{swarm}~\cite{li2010swarm}, 
\emph{group}~\cite{wang2006grouppattern} and \emph{platoon}~\cite{li2015platoon}. 
However, these solutions are tailored for each individual pattern and it is cumbersome to deploy and optimize each of the algorithm in real applications. Therefore, there 
calls for a general framework which provides versatile and efficient support of all these pattern discoveries. 

We observe that these co-movement patterns can be uniformly represented as a two-step neighborhood analytics as follows:
	
	(1) $\mathcal{N}_1$ is a \emph{distance neighborhood} used to determine the spatial proximity of objects. For example,  flock~\cite{gudmundsson2006computing} and group~\cite{wang2006grouppattern} patterns uses the \emph{disk-based} clustering, which is equivalent to $\mathcal{N}_1(o_i)= \{o_j | \mathtt{dist}(o_i,o_j) < r \}$ for each object. Convoy~\cite{jeung2008discovery}, swarm~\cite{li2010swarm} and platoon~\cite{li2015platoon} patterns uses the \emph{density-based} clustering, which is equivalent to $\mathcal{N}_1(o_i)= \{o_j | \mathtt{dist}(o_j,o_k) \leq \epsilon \wedge o_k \in \mathcal{N}_1(o_i)\}$.
	
	(2) $\mathcal{N}_2$ is a \emph{comparison neighborhood} used to determine co-moving behavior, i.e., $\mathcal{N}_2(o_i)=\{o_j, T | \forall t \in T, o_j \in C_t(o_i) \}$, where $C_t(\cdot)$ returns the neighborhoods (i.e., $N_1$) of an object at time $t$.

Enlightened by the neighborhood based unification, we propose a \emph{General Co-Movement Pattern} (GCMP)
query to capture all existing co-movement patterns in one shot. In GCMP, we treat the proximity detection (i.e., $\mathcal{N}_1$) as
a black box and only focus on the pattern detection (i.e., $\mathcal{N}_2$). By tuning different parameters (as explained in later sections), GCMP
query is able to detect any of the existing patterns.
%
%It is notable that the GCMP
%query is able to detect any of the existing co-movement patterns by adopting different analytic functions.



On the technical side, we study how to efficiently process GCMP query 
on the modern parallel platform (i.e., Apache Spark) to
gain scalability over large-scale trajectories. 
In particular, we propose two parallel frameworks: (1) TRPM, which partitions trajectories by replicating snapshots in the temporal domain. Within each partitions, a line-sweep method is developed to find all patterns. 
(2) SPARE, which partitions trajectories based on object's neighborhood. Within each partitions, a variant of Apriori enumerator is applied to generate all patterns. 
We deploy the two solutions in our in-house cluster with 11 machines. The experiments on three
real trajectory datasets upto 170 million data points confirms the scalability and efficiency 
of our methods.
%
%We then show the efficiency of both our methods in the Apache Spark platform with three real trajectory datasets upto 170 million points. The results show that SPARE achieves upto 14 times efficiency as compared to TRPM, and 112 times speedup as compared to the state-of-the-art centralized schemes. 


\section{Thesis Organization}
The rest of the thesis is organized as follows: in Chapter 2, we review the literature related
to our proposed queries in different data domains.
In Chapter 3, we present the Graph Window Query on graph data. 
In Chapter 4, we present the $k$-Sketch Query on sequence data.
In Chapter 5, we present the General Co-Movement Pattern Query on trajectory data. 
Chapter 6 summarizes this thesis and highlights future directions.

\section{Published Material}
The research in this thesis has led to numerous publications. 

\begin{itemize}
	\item{The overview of the thesis has been published in SIGMOD Ph.D. Symposium.\\
		\textbf{Qi Fan}, Kian-Lee Tan. Towards Neighborhood Analytics. Proceedings of the ACM SIGMOD on PhD Symposium, 2015
	}
	\item{The work in Chapter 3 appears in DASFAA 2016.\\
		\textbf{Qi Fan}, Zhengkui Wang, Chee-Yong Chan, Kian-Lee Tan. Towards Window Analytics over Large-Scale Graphs. International Conference on Database Systems for Advanced Applications, 2016
	}
	\item{The finding in Chapter 4 have been submitted for publication.\\
		\textbf{Qi Fan}, Yuchen Li, Dongxiang Zhang, Kian-Lee Tan. Discovering Newsworthy Themes From Sequenced Data: A Step Towards Computational Journalism.
	}
	\item{The work in Chapter 5 has been accepted in VLDB 2017.\\
		\textbf{Qi Fan}, Dongxiang Zhang, Huayu Wu, Kian-Lee Tan. A General and Parallel Platform for Mining Co-Movement Patterns over Large-scale Trajectories. Proceedings of the VLDB Endowment, 2017
	}	
\end{itemize}

%The overview of the thesis has been published in:
%
%Qi Fan, Kian-Lee Tan. ``Towards Neighborhood Analytics.'' \emph{Proceedings of the ACM SIGMOD on PhD Symposium}, 2015
%
%The work in Chapter 3 appears in:
%
%Qi Fan, Zhengkui Wang, Chee-Yong Chan, Kian-Lee Tan. ``Towards Window Analytics over Large-Scale Graphs.'' \emph{International Conference on Database Systems for Advanced Applications}, 2016
%
%The findings in Chapter 4 have been submitted for publication.
%
%The work in Chapter 5 has been accepted in:
%
%Qi Fan, Dongxiang Zhang, Huayu Wu, Kian-Lee Tan. ``A General and Parallel Platform for Mining CoMovement Patterns over Largescale Trajectories.'' \emph{Proceedings of the VLDB Endowment}, 2017