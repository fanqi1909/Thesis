\section{Contributions}
%This thesis strives to showcase the usefulness of the neighborhood
%concepts and address the efficiency issues when adopting with
%nontrivial analytic functions. 
In brief, this thesis entitles a twofold contribution.
First, by sewing different $\mathcal{N}$s and $\mathcal{F}$s, 
several novel analytic queries are proposed for 
\emph{graph}, \emph{sequence data} and \emph{trajectory} domains respectively. 
%three interesting
%neighborhood analytic queries are proposed for \emph{graph}, \emph{time series} 
%and \emph{trajectory} domains respectively. 
Second, this thesis
deals with the efficiency issues in deploying the corresponding analytic queries to
handle data of nowadays scale.
The roadmap of this thesis is shown in Figure~\ref{fig:thesis_roadmap}.
\begin{figure}[h]
\centering
\includegraphics[width=0.8\linewidth]{thesis_roadmap.pdf}
\caption{The road map of this thesis. There are three major contributions as highlighted in the center. Each contribution
is a neighborhood analytic based on different $\mathcal{N}$ and $\mathcal{F}$ as indicated by arrows.} 
\label{fig:thesis_roadmap}
\end{figure}

In a nutshell, we propose and enrich the neighborhood analytics in three data domains. 
In \emph{graph}, we identify two types
of window queries: \emph{k-hop} window is based on the \emph{distance} neighborhood
and \emph{topological} window is based on the \emph{comparison} neighborhood. 
In \emph{sequence data}, we leverage the nested \emph{distant} and \emph{comparison} neighborhoods 
to design the $k$-sketch query in finding prominent rank-aware streaks.
In \emph{trajectory}, we analyze existing movement patterns based
on \emph{distance} and \emph{comparison} neighborhoods. Then we propose a general query
to capture movement patterns built on top of the \emph{comparison} neighborhoods.

\subsection{Window Queries for Graph Data}
The first piece of the thesis deals with neighborhood query
on graph data. Nowadays information network are typically
modeled as attributed graphs where the 
vertexes correspond to objects and the edges capture the
relationships between these objects. As vertexes embed a wealth
of information (e.g., user profiles in social networks), there are 
emerging demands on analyzing these data to extract useful insights. 
We propose the concept of \emph{window analytics} 
for attributed graph and identify two types of such analytics as shown in the following examples:

\begin{example}[$k$-hop window]
In a social network (such as Linked-In and Facebook etc.), users are normally modeled as vertexes and connectivity relationships are modeled as edges. In the social network scenario, it is of great interest to summarize the most relevant connections to each user such as the neighbors within $2$-hops. Some analytic queries such as summarizing the related connections' distribution among different companies, and computing age distribution of the related friends can be useful. In order to answer these queries, collecting data from every user's neighborhoods within 2-hop is necessary.
\end{example}
\begin{example}[Topological window]
In biological networks (such as Argocyc, Ecocyc etc.), genes, enzymes and proteins are vertexes and their
dependencies in a pathway are edges. Because these networks are directed and acyclic, 
in order to study the protein regulating process, one may be interested to find out the statistics of molecules in each protein production pathway. For each protein,we can traverse the 
graph to find every other molecules that are in the upstream of its pathway.
Then we can group and count the number of genes and enzymes among those molecules.
\end{example}

The two \emph{windows} shown in the examples are essentially neighborhood functions defined for each vertex. Specifically, let $G=(V:E:A)$ be an attributed graph, where $V$ is the set of vertexes, $E$ is the set of edges, and each vertex $v$ is associated as a multidimensional points $a_v \in A$ called attributes.
The \emph{k-hop} window is a \emph{distance} neighborhood function, 
i.e., $\mathcal{N}_1(v,k)= \{u|\mathtt{dist}(v,u) \leq k\}$, 
which captures the vertexes that are $k$-hop nearby. 
The \emph{topological} window,  $\mathcal{N}_2(v)= \{u | u \in v.ancestor\}$,
is a \emph{comparison} neighborhood function that captures
the ancestors of a vertex in a directly acyclic graph.  The analytic function $\mathcal{F}$ is an aggregate function (sum, avg, etc.) on $A$.

Apart from demonstrating the useful use-cases on these two windows, 
we also investigate on supporting efficient window processing.  We propose
two different types of indexes: Dense Block Index (DBIndex)
and Inheritance Index (I-Index). The DBIndex and I-Index
are specially optimized to support k-hop window and topological
window processing. These indexes
integrate the aggregation process with partial work sharing techniques
to achieve efficient computation.
In addition, we develop space-and-performance efficient techniques
for the index construction. Notably, DBIndex saves upto 80\%
of indexing time as compared to the state-of-the-art competitor and upto 10x
speedup in query processing. 


\subsection{$k$-Sketch Query for Sequence Data}
%Rank-aware Streak Discovery in Time Series}
The second piece of the thesis explores the neighborhood
query for sequence data. 
In sequence data analysis, an important
and revenue-generating task is to detect phenomenal patterns.
An outstanding neighborhood based pattern is the \emph{streak} which are constructed by aggregating
temporally nearby data points for each subject.
%
%each \emph{streak} is an analytic function applied on
%a subsequence of data.
%
% A \emph{streak} is constructed by treating
%each temporally nearby data points
%
%for a data point is constructed by combing
%data points that are temporally closed to it. Effectively, each
%resulting \emph{streak} is a consecutive subsequence of the original sequence. Then
%analytic functions are applied to each \emph{streak}.
%
%
%An outstanding example of such patterns is the \emph{streak}
%which refers to an analytics applied on a consecutive portion of a sequence. 
Streak has been found useful in many time series applications,
such as network monitoring, stock analysis
and computational journalism. 
However, the amount of streaks
is too huge (i.e., $O{N \choose 2}$ for a subject with $N$ data points) 
to conduct effective analytics. Observed that
many streaks share almost identical information, we are motivated 
to propose a $k$-Sketch
query to effectively select $k$ most representative streaks.

The $k$-sketch query is built on a core concept named \emph{rank-aware streak},
out of which $k$ representatives are 
chosen based on a novel scoring function
that balances the strikingness and the diversity. A \emph{rank-aware streak}
enriches traditional streak by including the
relative position of a streak among its cohort.
Such a concept is not rare in real-life.
For example, journalists often adopt the rank-aware streak to promote
the attractiveness of their news:

% For each subject,
%we then select $k$ \emph{rank-aware} streaks that best cover all
%data points of a subject with highest ranks called $k$-sketch.
%In real life,
%rank-aware streaks are often seen in journalism to promote the attractiveness
%of news.
%For example:
%We notice that many streaks share the same information which
%result in near-duplicate output. Therefore, we propose a $k$-Sketch
%query to effectively select $k$ most representative streaks.
%In modeling, we first enrich the semantic of stream 
%%We enrich the semantic of \emph{streak} 
%by proposing the \emph{rank-aware} streak, which computes the
%relative position of a streak among its cohort. In real-life,
%rank-aware streaks are often adopted to promote the attractiveness of news. For example:

\eat{The second piece of the thesis proposes a neighborhood
query in \emph{time series} domain to support the rank-aware streak discovery. 
In time series domain, an 
important and revenue generating tasks is to detect phenomenal patterns. However,
as the large amount of the data stored and continuously generated, it is
almost impossible for humans to manually detect these patterns. A prominent 
example of such patterns is \emph{streak}, which refers to a subject is outstanding
for consecutive events. We enhance the traditional \emph{streak} to incorporate with
rank-awareness. Coincidentally, the \emph{rank-aware streak} are very prevalent
in real life:}
\eat{
The second piece of the thesis proposes a neighborhood
query to support automatic discovery of news in time series data.
Nowadays events are collected in a timestamped format and journalists needs to manually synthesize those events to produce interesting news. We propose the automatic \emph{rank-aware news theme} detection with the aim to alleviate human efforts in poring over a large amount of data. 
Coincidentally, the \emph{rank-aware news theme}s are very prevalent
in real life new reporting:}
\begin{enumerate}
\setlength\itemsep{-0.05cm}
\item(Feb 26, 2003) With 32 points, Kobe Bryant saw his 40+ scoring streak end at \textbf{nine} games,  tied with Michael Jordan for \textbf{fourth} place on the all-time list\footnote{\url{http://www.nba.com/features/kobe_40plus_030221.html}}. 

\item(April 14, 2014) Stephen Curry has made 602 3-pointer attempts from beyond the arc,... are the \textbf{10th} most in NBA history in a season (\textbf{82 games})\footnote{\url{http://www.cbssports.com/nba/eye-on-basketball/24525914/stephen-curry-makes-history-with-consecutive-seasons-of-250-3s}}.

\item (May 28, 2015) Stocks gained for the \textbf{seventh consecutive day} on Wednesday as the benchmark moved close to the 5,000 mark for \textbf{the first} time in seven years\footnote{\url{http://www.zacks.com/stock/news/176469/china-stock-roundup-ctrip-buys-elong-stake-trina-solar-beats-estimates}}.

\item (Jun 9,  2014) Delhi has been witnessing a spell of hot weather  over the \textbf{past month}, with temperature hovering around 45 degrees Celsius, .... \textbf{highest} ever since 1952\footnote{\url{http://www.dnaindia.com/delhi/report-delhi-records-highest-temperature-in-62-years-1994332}}.

\item(Jul 22, 2011) Pelican Point recorded a maximum rainfall of 0.32 inches for \textbf{12 months}, making it the  \textbf{9th driest} places on earth\footnote{\url{http://www.livescience.com/30627-10-driest-places-on-earth.html}}.
\end{enumerate}
 
In the above examples, each news is a rank-aware streak consisting of five elements: a subject (e.g., Kobe Bryant, Stocks, Delhi), an event window (e.g., nine straight games, seventh consecutive days, past month), an aggregate function on an attribute
(e.g., minimum points, count of gains, average of degrees), a rank (e.g., fourth, first time, highest), and a
historical dataset (e.g., all time list, seven years, since 1952). These indicators are summarized in
Table~\ref{tbl:news-example}.

The rank-aware streak can be formally described using a joint neighborhood function. 
Let $e_s(t)$ denote the event of subject $s$ at time $t$. Then the rank-aware streaks are generated using neighborhood analytics in a two-step manner: 

(1) a \emph{distance neighborhood} $\mathcal{N}_1(o_i,w)=\{o_j | o_i.t - o_j.t \leq w \}$ groups a consecutive $w$ events for each event. Let $\overline{v}$ be the aggregate value associated with $\mathcal{N}_1$, then the output of this step is a set of \emph{event windows} of the form  $n=\langle o_i, w, t, \overline{v} \rangle$.

(2) a \emph{comparison neighborhood} $\mathcal{N}_2(n_i) = \{n_j | n_j.w = n_i.w \wedge n_i.\overline{v} \geq n_j.\overline{v} \}$ ranks a subject's event window among all other event windows with the same window size. The result of the this step is a tuple $\langle o_i, w, t, r \rangle$, where $r$ is the \emph{rank}.



\eat{In the above news themes, there are a subject (e.g., Kobe Bryant, Stocks, Delhi), an event window (e.g., nine straight games, seventh consecutive days, past month), an aggregate function on an attribute
(e.g., minimum points, count of gains, average of degrees), a rank (e.g., fourth, first time, highest), and a
historical dataset (e.g., all time list, seven years, since 1952). These news theme indicators are summarized in Table~\ref{tbl:news-example}.  }


\begin{table}[h]
\centering
\begin{tabular}{|c|c|c|c|c|}
\hline
\textbf{E.g.} & \textbf{Subject} & \textbf{Aggregate function} & \textbf{Event window} & \textbf{Rank} \\
\hline
1 & Kobe Bryant &$\mathtt{min}$(points) & 9 straight games & 4 \\
\hline
2 & Stephen Curry &$\mathtt{sum}$(shot attempts) & 82 games & 10 \\
\hline
3 & Stocks &$\mathtt{count}$(gains) & 7 consecutive days & 1 \\
\hline
4 & Delhi &$\mathtt{avg}$(degree) & past months (30 days) & 1 \\
\hline
5 & Pelican Point &$\mathtt{max}$(raindrops) & 12 months & 9 \\
\hline
\end{tabular}
\caption{News theme summary}
\label{tbl:news-example}
\end{table}

%\revised{We model these rank-aware streaks using nested neighborhood analytics.}
\eat{We model these news themes using nested neighborhood analytics.
Let $e_s(t)$ denote the event of subject $s$ at time $t$. Then the rank-aware streaks are generated using neighborhood analytics in a two-step manner: 

(1) a \emph{distance neighborhood} $\mathcal{N}_1(o_i,w)=\{o_j | o_i.t - o_j.t \leq w \}$ groups a consecutive $w$ events for each event. Let $\overline{v}$ be the aggregate value associated with $\mathcal{N}_1$, then the output of this step is a set of \emph{event windows} of the form  $n=\langle o_i, w, t, \overline{v} \rangle$.

(2) a \emph{comparison neighborhood} $\mathcal{N}_2(n_i) = \{n_j | n_j.w = n_i.w \wedge n_i.\overline{v} \geq n_j.\overline{v} \}$ ranks a subject's event window among all other event windows with the same window size. The result of the this step is a tuple $\langle o_i, w, t, r \rangle$, where $r$ is the \emph{rank}.}

%It is easy to verify that $\mathcal{N}_2 \cdot \mathcal{N}_1$ result in the rank-aware streaks.
% is the \emph{rank-aware news themes}. Rooted from the \emph{rank-aware news themes}, we further address the problem 
%of controlling the result size.
%We propose
%a novel concept named \emph{Sketch} to avoid outputting near-duplicate themes.
%A sketch contains $k$ most representative rank-aware news themes under a scoring function that considers both strikingness and diversity.  Our objective is to discover sketches for each subject in the domain.

Besides proposing the rank-aware streak, we also technically 
study how to efficiently support the \emph{$k$-Sketch} query in both
offline and online scenarios. We propose various window-level 
pruning techniques to find striking candidate steaks.
Among those candidates, we then develop approximation
methods, with theoretical bounds, to discover $k$-sketches for each subject. 
We conduct experiments on four real
datasets, and the results demonstrate the efficiency and 
effectiveness of our proposed algorithms: the running time
achieves up to 500x speedup as compared to baseline and the quality of the
detected sketch is endorsed by the anonymous users
from Amazon Mechanical Turk \footnote{\url{https://requester.mturk.com}}.

\subsection{Co-Movement Pattern Query in Trajectory Data}
The third piece of the thesis studies the neighborhood
query on trajectory data. An important mining task
in trajectory domain is to discover the movement patterns
of objects which is useful in a wide spectrum of applications.
%
%In the trajectory domain, neighborhood
%functions can effectively model the movement patterns
%of objects which is useful in a wide
%spectrum of applications. 
%
%An important mining task in trajectory
%data that has drawn many research attention is discovering
%movement patterns. Coincidentally, a movement pattern
%can be 
%
%
%Discovering co-movement patterns from large-scale trajectory 
%databases is an important mining task and has a wide
%spectrum of applications. 
Existing studies have identified several types of interesting movement patterns called \emph{co-movement} patterns. A co-movement pattern refers to a group of moving objects traveling together for a certain period of time and the group of objects is normally determined by their spatial proximity. A pattern is prominent if the group size exceeds $M$ and the length of duration exceeds $K$. 
Inspired by the basic definition 
and driven by different mining applications, there are a bunch of variant 
co-movement patterns that have been developed with more advanced constraints.

\begin{figure}[h]
\centering
\includegraphics[width=0.8\textwidth]{trajectory_patterns.pdf}
\caption{Trajectories and co-movement patterns. The example consists of six trajectories across six snapshots. Objects in spatial clusters are enclosed by dotted circles. $M$ is the minimum cluster cardinality; $K$ denotes the minimum number of snapshots for the occurrence of a spatial cluster; and $L$ denotes the minimum length for local consecutiveness.}
\label{fig:related_work}
\end{figure}

Figure~\ref{fig:related_work} is an example to demonstrate the concepts of various co-movement patterns. The trajectory database consists of six moving objects and the temporal dimension is discretized into six snapshots. In each snapshot, we treat the clustering method as a black-box and assume that they generate the same clusters. Objects in proximity are grouped in the dotted circles. As aforementioned, there are three parameters to determine the co-movement patterns and the default settings in this example are $M=2$, $K=3$ and $L=2$. Both the \emph{flock} and the \emph{convoy} require the spatial clusters to last for at least $K$ consecutive  timestamps. Hence,$\langle o_3,o_4:1,2,3 \rangle$ and $\langle o_5,o_6:3,4,5 \rangle$  remains the only two candidates matching the patterns. The \textit{swarm} relaxes the pattern matching by discarding the temporal consecutiveness constraint. Thus, it generates many more candidates than the \textit{flock} and the \textit{convoy}. The \textit{group} and the \textit{platoon} add another constraint on local consecutiveness to retain meaningful patterns. For instance, $\langle o_1,o_2:1,2,4,5 \rangle$ is a pattern matching local consecutiveness because timestamps $(1,2)$ and $(4,5)$ are two segments with length no smaller than $L=2$. The difference between the \textit{group} and the \textit{platoon} is that the \textit{platoon} has an additional parameter $K$ to specify the minimum number of snapshots for the spatial clusters. This explains why $\langle o_3,o_4,o_5:2,3 \rangle$ is a \textit{group} pattern but not a \textit{platoon} pattern.


These co-movement patterns can be uniformly described as a two-step neighborhood query as follows:
	(1) $\mathcal{N}_1$ is a \emph{distance neighborhood} used to determine the spatial proximity of objects. For example,  flock and group patterns uses the \emph{disk-based} clustering, which is equivalent to $\mathcal{N}_1(o_i)= \{o_j | \mathtt{dist}(o_i,o_j) < r \}$ for each object. Convoy, swarm and platoon patterns uses the \emph{density-based} clustering, which is equivalent to $\mathcal{N}_1(o_i)= \{o_j | \mathtt{dist}(o_j,o_k) \leq \epsilon \wedge o_k \in \mathcal{N}_1(o_i)\}$.
	(2) $\mathcal{N}_2$ is a comparison neighborhood used to determine the temporal constraints, i.e., $\mathcal{N}_2(o_i)=\{o_j, T | \forall t \in T, C_t(o_i) = C_t(o_j)\}$, where $C_t(\cdot)$ returns the cluster ID of an object at time $t$.

Based on the neighborhood unification, we propose a \emph{General Co-Movement Pattern} (GCMP)
query to capture all existing co-movement patterns. In GCMP, we treat the proximity detection (i.e., $\mathcal{N}_1$) as
a black box and only focus on the pattern detection (i.e., $\mathcal{N}_2$). It is notable that the GCMP
query is able to detect any of the existing co-movement patterns by adopting different analytic functions.


%We notice that these patterns can be unified using \emph{neighborhood} query.
%Let $C_t(o_i)$ be the cluster which $o_i$ belongs to at snapshot $t$, then a co-movement pattern 
%can be viewed as a \emph{comparison neighborhood} $\mathcal{N}(o_i)=\{o_j, T | \exists t \in T, C_t(o_i) = C_t(o_j)\}$. For each generated neighborhood, the analytic function is the pattern discovery
%based on temporal constraints. Therefore, we propose a \emph{General Co-Movement Pattern} (GCMP) based on such neighborhood to capture all existing co-movement patterns. The GCMP can be reduced to any of the existing co-movement patterns by adapting with different analytic functions. Note that detection of spatial proximity of objects itself can be viewed as a neighborhood query. For example, to discover the \emph{flock} pattern, the $\mathcal{N}_f(o_i)=\{o_j, T | \mathtt{dist}(o_i,o_j) < r\}
%$ is used to find a region of objects within radius $r$. To facilitate the purpose general comovement pattern, we assume
%the spatial proximity of objects are given and focus on the second neighborhood function.


Technical wise, we study how to efficiently processing GCMP in a MapReduce platform to gain scalability for large-scale trajectory databases. We propose two parallel frameworks: (1) TRPM, which partitions trajectories by replicating snapshots in the temporal domain. Within each partitions, a line-sweep method is developed to find all patterns. (2) SPARE, which partitions trajectories based on object's neighborhood. Within each partitions, a variant of Apriori enumerator is applied to generate all patterns. We then show the efficiency of both our methods in the Apache Spark platform with three real trajectory datasets upto 170 million points. The results show that SPARE achieves upto 14 times efficiency as compared to TRPM, and 112 times speedup as compared to the state-of-the-art centralized schemes.