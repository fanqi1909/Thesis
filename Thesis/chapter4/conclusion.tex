\section{Conclusion and Future works} \label{sec:conclusion}
In this paper, we look at the problem of automatically discovering striking news themes from sequenced data. 
We group neighbourhood events into windows and propose a novel idea of \emph{ranking} to represent candidate news themes. We then formulate the \emph{sketch discovery} problem which aims to select the $k$ best new themes as the sketch for each subject. 
We study the sketch discovery problem in both offline and online scenarios,
and propose efficient solutions to cope each scenario. In particular, we design novel window-level pruning techniques and a (1-1/e) greedy algorithm to achieve efficient sketch discovery in offline. Then we design a 1/8 approximation algorithm for the online sketch discovery.
Our comprehensive experiments demonstrate the efficiency of our solutions and a human study confirms the effectiveness of sketch discovery in news reporting.

Our work opens a wide area of research in computational journalism. In our next step, we aim to extend the event sources from sensor data to non-schema data such as tweets in social networks. We also aim to investigate multi-subject models to automatically generate news themes across different subjects.


