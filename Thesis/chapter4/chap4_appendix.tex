\section{Discussions on Other Aggregate Functions}
\label{sec:discussion}
First, we shall see that supporting \emph{sum} is equivalent
to supporting \emph{avg}. 
A candidate theme which has a rank under \emph{avg} will have the 
same rank under \emph{sum} as the ranking is derived by comparing all 
candidates with the same window length. 
%
Second, supporting \emph{count} is equivalent to supporting \emph{sum}. 
By assigning each event with value of either 1 or 0, we can apply the 
same pruning bounds for \emph{sum} to support \emph{count}. 
%
Third, supporting \emph{max} is equivalent to supporting \emph{min}. This is 
because when \emph{max} is used as the aggregate function, we are more interested 
to find news themes which have smaller aggregation values. 
For example, ``XXX stock has a maximum of \$0.2 price
in consecutive 10 days, which is the lowest ever''. Then finding the sketches according to \emph{max} 
can be derived from \emph{min} directly by negating the event values. 
Therefore, we only provide bounds for 
\emph{sum} and \emph{min}, which are shown as in Table~\ref{tbl:agg_bound}:
%
{\renewcommand{\arraystretch}{1.2} 
\begin{table}[h]
\caption{Bounds for other aggregate functions}
\centering
\begin{tabular}{|c|c|}
\hline 
\textbf{Aggregate Function} & \textbf{Subadditivity} \\
\hline 
\emph{sum} & $J_s(w) \leq J_s(w_1) + J_s(w-w_1) $ \\
\emph{min} & $J_s(w) \leq max(J_s(w_1), J_s(w-w_1))$ \\
\hline 
\textbf{Aggregate Function} & \textbf{Visting-Window Bound} \\
\hline 
\emph{sum} & $J_s(w) = J_s(w-1)+J_s(1)$ \\
\emph{min} & $J_s(w) = J_s(w/2)$ \\
\hline 
\textbf{Aggregate Function}& \textbf{Unseen-Window Bound} \\
\hline 
\emph{sum} & $M_s(w) = W_s(t,w).\overline{v} + J_s(t-w)$ \\
\emph{min} & $M_s(w) = max\{W_s(t,w).\overline{v}, J(1)\}$ \\
\hline 
\textbf{Aggregate Function }& \textbf{Online-Window Bound} \\
\hline 
\emph{sum} & $M_s(w) = W_s(t,w).\overline{v} + J_s(t-w)$ \\
\emph{min} & $M_s(w) = max\{W_s(t,w).\overline{v}, J(1)\}$ \\
\hline 
\end{tabular} 
\label{tbl:agg_bound}
\end{table}
}

%In fact, all distributive aggregation functions can leverage similar 
%technics to support efficient pruning. 


%\begin{itemize}
%\item \emph{sum}: $J_s(w) \leq J_s(w_1) + J_s(w-w_1) $
%\item \emph{min}: $J_s(w) \leq max(J_s(w_1), J_s(w-w_1))$
%\end{itemize}
%\noindent\textbf{Visting-Window Bound}
%\begin{itemize}
%\item \emph{sum}: $J_s(w) = J_s(w-1)+J_s(1)$
%\item \emph{min}: $J_s(w) = J_s(w/2)$
%\end{itemize} 
%\noindent\textbf{Unseen-Window Bound}
%\begin{itemize}
%\item \emph{sum}:$M_s(w) =\lfloor \mathbb{H}_s /w\rfloor J_s(w) + J_s(w-1)$
%\item \emph{min}: $M_s(w) = max\{J(w), J(1)\}$
%\end{itemize}
%\noindent\textbf{Online-Window Bound}
%\begin{itemize}
%\item \emph{sum}:$M_s(w) = W_s(t,w).\overline{v} + J_s(t-w)$
%\item \emph{min}:$M_s(w) = max\{W_s(t,w).\overline{v}, J(1)\}$
%\end{itemize}

