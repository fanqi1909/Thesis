\section{Problem Formulation}\label{gw:sec:pf}
In this section, we provide the formal definition of graph window query.
%In this section, we provide the formal definition of graph window query. 
We use $G = (V,E)$ to denote a directed/undirected data graph, where $V$ is its vertex set and $E$ is its edge set.
Each node/edge is associated with a (possibly empty) set of attribute-value pairs.

%\begin{defn}
%\label{attributed_graph}
%[Attributed Graph]
%Let $G = (V,E,A)$ be an attribute graph, where $V$ is  its vertex set, $E$ is its edge set and $A$ is its attribute set respectively. An edge $e(u,v) \in E$ indicates that $u$ is linked to $v$, where $u \in V$ and $v \in V$. For every vertex in the graph, there exists one multidimensional vector in $A$ representing the attributes of the vertex. i.e. $\forall v \in V, \exists a(v)=(a_1,a_2..,a_k) \in A$.
%\end{defn} 

Figure~\ref{fig:attributed} (a) shows an undirected graph representing 
a social network that will be used as our running example in this chapter. 
For convenience, each vertex is labeled by its ``user'' attribute;
and there is one edge between vertex X and vertex Y if user X and user Y are connected in the social network.
The table in Figure~\ref{fig:attributed} (b) shows the values of five attributes (User, Age, Gender, Industry, and Number of posts) associated with each user.


\begin{figure}[h]
\centering
\includegraphics[width=0.8\textwidth]{chapter3/attributed_graph.pdf}
	\caption{A miniature social graph. (a) the graph structure. (b) the attributes associated with the vertexes in (a).} 
	\label{fig:attributed}
\end{figure}

Given a data graph $G = (V,E)$,
a \emph{Graph Window Function (GWF)} over $G$ can be expressed 
as a pair $(W, \Sigma)$, where 
$W(v)$ denotes a \emph{window specification} for a vertex $v \in V$ 
that determines the set of $v$'s neighborhood nodes\footnote{We use ``vertex'' to refer the vertex in the original graph and ``node'' to refer to the vertex in the windows.},
$\Sigma$ denotes an \emph{aggregate function}\footnote{An aggregate function is associated with one attribute. I.e., $\mathtt{average}$(age) and $\mathtt{average}$(salary) are considered to be two different aggregate functions. Functions that associated with more attributes can be easily computed via stored views.}.
%, and $A$ denotes  a \emph{vertex attribute}.
The evaluation of a GWF $(W, \Sigma)$ on $G$
computes for each vertex $v$ in $G$, the aggregation $\Sigma$ 
%on the  values of attribute $A$  
over all the nodes in $W(v)$, which is denoted by $\Sigma_{v' \in W(v)} v'$.
%
In this chapter, we focus on the distributive or algebraic aggregate functions (e.g., sum, count, average), as these aggregate functions are widely used in practice. 
%In other words, $W(v)$ refers to a set of vertexes, and the aggregation function 
%$\Sigma$ operates on the values of attribute $A$ over all the vertexes in $W(v)$. 
%Meanwhile, the aggregate function $\Sigma$ is distributive or algebraic (e.g., sum, count, average), as these aggregate functions are widely used in practice. 
 
In the following, we introduce two useful types of window specification (i.e., $W$), namely, 
\emph{k-hop window} and \emph{topological window}.


\begin{definition}[$k$-hop Window] 
Given a vertex $v$ in a data graph $G$, 
the $k$-hop window of $v$, denoted by $W_{kh}(v)$ (or $W(v)$ when there is no ambiguity),
is the set of nodes in $G$ which can be reached by $v$ within $k$ hops.
For an undirected graph $G$,
a node $u$ is in $W_{kh}(v)$  if there is a $\alpha$-hop path between $u$ and $v$ where $\alpha \leqslant k$.
For a directed graph $G$,
a node $u$ is in $W_{kh}(v)$  if there is a $\alpha$-hop directed path from $v$ to $u$\footnote{
Other variants of k-hop window for directed graphs are possible; e.g.,
a node $u$ is in $W_{kh}(v)$  if there is a $\alpha$-hop directed path from $u$ to $v$ where $\alpha \leqslant k$.
} where $\alpha \leqslant k$.
\end{definition}

Intuitively, a $k$-hop window selects the neighboring nodes within a $k$-hop distance. 
These neighboring nodes typically represent the most important 
entities to a vertex with regard to their structural relationship in a graph. 
Thus, the $k$-hop window provides a meaningful specification for many applications, such as customer behavior analysis \cite{briscoe2013determining,dai2012predicting} , digital marketing \cite{ma2010ego} etc.
%\remark{Cite some papers here to justify.}

As an example, in Figure~\ref{fig:attributed}, the $1$-hop window of vertex \emph{E} is $\{A,C,E\}$ and the $2$-hop window of vertex \emph{E} is $\{A,B,C,D,E,F\}$.  

\begin{definition}[Topological Window] 
Given a vertex $v$ in a DAG $G$, the topological window of $v$, denoted by $W_t(v)$,
refers to the set of ancestor nodes of $v$ in $G$,
i.e., a vertex $u$ is in $W_t(v)$ if there is directed path from $u$ to $v$ in $G$.
\end{definition}

There are many directed acyclic graphs (DAGs) in real world applications (such as biological networks, citation networks and dependency networks)
where topological windows represent meaningful relationships that are of interest.
For example, in a citation network where (X,Y) is an edge if paper $X$ cites paper $Y$, 
the topological window of a paper represents the citation impact of that paper \cite{campanario2011empirical,holsapple2003citation,ma2008bringing}.

To illustrate, Figure~\ref{fig:topological} shows a small example of a Pathway Graph from a biological network. 
The topological window of $E$ (i.e., $W_t(E)$) is $\{A, B, C, D, E\}$ and $W_t(H)$ is $\{A, B, D, H\}$.


\begin{figure}[h]
\centering
 \includegraphics[width=0.8\textwidth]{chapter3/dag.pdf}
	\caption{A miniature pathway DAG. (a) the DAG structure. (b) the attributes associated with the vertexes of (a).}
	\label{fig:topological}
\end{figure}

\begin{definition}[Graph Window Query] 
A graph window query on a data graph $G$ is of the form
$GWQ(G, W_1, \Sigma_1,\cdots, W_m, \Sigma_m)$, where $m \geq 1$
and
each pair $(W_i,\Sigma_i)$ is a graph window function on $G$.
\end{definition}
In this chapter, we focus on graph window queries with a single window 
function that is either a $k$-hop or topological window. 
The evaluation of complex graph window queries with multiple window 
functions can be simply processed as a sequence of window functions one
after another. 
We leave the optimization of processing multiple window functions for future studies.

